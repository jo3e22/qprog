\documentclass[11pt,twoside,a4paper]{report}
%\setlength{\textwidth}{155.0mm}

%%======================================================================
%% PACKAGES:
%%
\usepackage{times}               % Times+Helvetica+Courier fonts
%\usepackage{helvet}              % helvetica + cmr
%\usepackage{frutiger}            % Frutiger \sffamily fonts.
\usepackage{fancyhdr}       % package for headers/footers
%\usepackage{culhamart}           % headings in \sffamily\bfseries
\usepackage{amsmath}
\usepackage{amssymb}
\usepackage{graphicx}            % Graphics.
%\usepackage{subfigure}           % Subfigures
\usepackage{a4}                  % page layout to fit A4
%\usepackage{amssymb}              % 
%\usepackage{lastpage}            % get page no of last page
%\usepackage{ifthen}              % logical branching
\usepackage{hyperref}            %insert hyper-links
\usepackage{latexsym}
% uncomment the following to override auto page total
%\pptotal{20}
%%======================================================================


%\newcommand{\culhamshorttitle}{CCFE-QPROG-WI1-Draft}%<==edit
\newcommand{\culhamshorttitle}{CCFE-R(15) 34}%<==edit
\newcommand{\Sec}[1]{Section~\ref{sec:#1}}
\newcommand{\Fig}[1]{Figure~\ref{fig:#1}}
\newcommand{\Eq}[1]{Equation~(\ref{eq:#1})}
\newcommand{\Eqs}[2]{Equations(\ref{eq:#1}) and~(\ref{eq:#2})}
\newcommand{\Figs}[2]{Figures~\ref{fig:#1}--~\ref{fig:#2}}
%Bold for vectors, tt for computer names
\newcommand{\V}[1]{\mathbf{#1}}
\newcommand{\T}[1]{{\tt #1}}
\newcommand{\F}[1]{\mathcal{#1}}
\def\zetab{\mbox{\bf $\zeta$}}
\def\coden{QPROG} %<==edit


%%======================================================================
%% REPORT TITLE PAGE Information

\title {\culhamshorttitle \\ \vspace*{0.5cm}
CCFE\thanks{
UK Atomic Energy Authority,
Culham Science Centre, Abingdon, Oxfordshire OX14 3DB
}
\\ \vspace*{0.5cm}
Fortran 95 Programming Style
}%<==edit





\newcommand{\culhamcontacttel}{Telephone: +44 1235 466433}%<==edit
\newcommand{\culhamcontactfax}{E-mail: wayne.arter@ccfe.ac.uk}%<==edit
\newcommand{\culhamcontactname}{Wayne Arter}%<==edit

%%qa box information -- change following as needed
\date{\today\\ \vspace*{5.0cm}
\copyright UKAEA 2015\\ \vspace*{1.0cm}
}
\author{\culhamcontactname \thanks{\culhamcontacttel \ \culhamcontactfax},
Nick Brealey\thanks{Cobham Technical Services, Network House, Langford Locks, Kidlington, OX5 1LH},
James Eastwood and Guy Morgan\thanks{Culham Electromagnetics Ltd, Culham Science Centre, Abingdon, OX14 3DB}}

\begin{document}
\maketitle

\tableofcontents
\cleardoublepage

\setlength{\parindent}{0pt}
\addtolength{\parskip}{0.5\baselineskip}
\setlength{\topsep}{0pt}
\setlength{\itemsep}{0pt}


%\pacs{52.35.-g, 47.55.P-, 02.30.Oz}

%\maketitle

%
%%%--------------------------------------------------------------------
%%  Abstract
%%
\begin{abstract}
%This report describes work under the \coden\ Project.
\noindent This document summarises the Fortran 95 programming style to be used
for object-oriented Fortran codes.
%for the \coden\ project.
\end{abstract}
%
%%====================================================================
%% BODY OF THE REPORT
%%---------------------------------------------------------------------

%---------------------------------------------------------------------
%                      The body of the report
%
\chapter{Introduction}\label{sec:intro}

The consistent use of a set of programming conventions
can make software easier to understand, develop  and maintain.
This leads to increased reliability of the software.

This work instruction (based on the unpublished report~\cite{Ea09Fort})
describes the programming conventions
for the use of Fortran~95 in the \coden\ project.
The programming conventions used have an impact on the following
aspects of the software which should be considered when making choices:
\begin{enumerate}
\setlength{\parsep}{0ex}
\setlength{\itemsep}{0ex}
\item clarity,
\item modularity,
\item defensive programming,
\item flexibility,
\item portability,
\item performance,
\item robustness and
\item traceability.
\end{enumerate}

By clarity we mean how easy the code is to read and understand.
Code layout, naming conventions and comments are low-level factors
which are particularly important here but the architecture of the
software and external documentation are also important.

Modularity is related to clarity. It means breaking the code up
into manageable pieces so that a developer can work on one piece of the
code without having a complete understanding of every other part of
the code.

By defensive programming we mean how robust the code is with respect
to programming errors as distinct from how the code handles errors due
to incorrect user input or system errors, which we regard as being
related to robustness.  We consider the code to be programmed
defensively if programming errors will stop the code
compiling. Factors which contribute to a defensive programming style
include using explicit interfaces to subroutines and assumed shape
arrays so that mismatched argument lists are detected, using keywords
in subroutine calls, using \verb|implicit none| and \verb|private| so
that mistyped variable names are more likely to be detected, using
derived types etc.

By flexibility we mean how easy it is to change the code. Leaving our
options open is part of this. Flexibility can be improved by using
derived types where we can change the implementation later, 
using \verb|private| so that we know that changing one part of the
code won't have an unexpected effect on another part of the code.

Portability means that the software should not be tied to particular
computers or compilers. For \coden, we aim to achieve portability by
writing software which conforms to the Fortran 95
standard~\cite{ISO1539}. There are however minor portability and
standards-compliance issues with obtaining command line arguments.
(Features from Fortran 2003 address these issues.)

Performance is concerned with how fast the code runs, how large a
problem can be run on a particular system and how accurate the results
are. The choice of algorithm and input parameters are the primary
factors here but choices such as using pointers or indices, or using derived
types or multiple arrays, may have an effect on performance.

Robustness is concerned with how the code behaves in response to user
errors or system errors such as not being able to allocate more memory
or running out of space on a disk. Checking the status of
\verb|allocate|, \verb|read| and \verb|write| statements and avoiding
large automatic arrays are relevant here.

Traceability is concerned with knowing which versions of the source 
files were used to build the software, which compiler was used,
what compiler options were used, whether the validation test deck been
run with a particular executable etc.

\chapter{Style Recommendations}
\label{sec:style}

This section describes recommendations for programming style.
Section~\ref{sec:arch} describes recommendations which are more
related to architecture, although the distinction may be subjective.

\section{Free source format}
Use the free source form which was introduced in Fortran 90.

\section{One module per file}
Only put one module in each file. We expect the program to consist
entirely of Fortran modules and one program unit. Interface modules
shall be used to provide interfaces to heritage Fortran 77 program units.

\section{File names}

The file name should be the same as the name of the module or 
program unit, in lower case, with the extension \verb|.f90|.

If it is necessary to have different versions of a file add a suffix
to the root part of the name e.g.\ if
we have a standard \verb|types| module in file \verb|types.f90| and
we want an alternative version which uses 32 bit numbers instead
of 64 bit  numbers we could put the alternative version in the
file \verb|types.r4.f90|.

{\it The \verb|.f90| means the free source format which was
  introduced in Fortran 90. Some people have suggested that we
   use  \verb|.f95| to indicate that the source is Fortran 95. The
   problem with this is that there are Fortran 
   95 compilers which do not accept the  \verb|.f95| extension.
   We use lower case for file names because they are easier 
   to read than uppercase.}


\section{Lower-case tokens}

Use lower case for all Fortran tokens (keywords, names, constants,
operators, edit descriptors etc) except for character literal constants.
(Comments should be in mixed case.)
   


\section{Naming conventions}


We advocate using the module name as the prefix for public variables,
public derived types and public routines in a module e.g.\ public
variables and routines in module \verb|group| should have names like
\verb|group_number| or \verb|group_first()|. 



It should be remembered that dummy argument names can be used as keyword 
names in Fortran 95 routines with explicit interfaces. Dummy argument names 
should be simple names which make sense when used as keyword arguments
e.g.\
\begin{verbatim}
   n_tree = tree_size( root_node=node(1) )
\end{verbatim}


Local variables and types should not have the module prefix. They should have
names like \verb|j| or \verb|max_iterations| (assuming that there is 
no \verb|max| module).

The names for private module variables and routines should be chosen
by the programmer to aid readability. They should not use the module
name as a prefix, so that they are distinguished from their public
counterparts.

Derived types should have the suffix \verb|_t| e.g.\ 
\verb|tree_node_t|. This allows the use of dummy argument names
without the \verb|_t| which is convenient for keyword arguments e.g.\ 
\verb|call tree_clear(tree_node=x)|.

\section{Object-oriented design pattern}
\label{sec:oo}

\coden\ will use an object-oriented design pattern to allow
multiple instances of various classes of object.

In this design pattern the data which represent an instance of an
object are stored in a variable of a derived type which is created
outside the module. The `member functions' which operate on the
object are module routines which take the data derived type 
as an argument. The derived type is defined in the module.

For this design pattern we need conventions for the module name,
the type which contains the data for an object, the constructor
for the object, the destructor for the object and the dummy argument
name and position for the variable containing the object's data.

The naming convention for \coden\ is 
\begin{enumerate}
\item The name with a \verb|_m| suffix is the name of the
  module (and the base name of the file containing the module).
\item The name with the suffix \verb|_t| is the derived type
  containing the data for the instance of an object.
\item \verb|self| is the name of the dummy argument of the
  module data type. The \verb|self| argument should be the
  first argument.
\item The constructor for an object should
  be a subroutine with the module name with the suffix \verb|_init|.
\item The destructor for an object should be a subroutine with the
  module name with the suffix \verb|_delete|.
\item When a procedure needs to have an object of
  type \verb|xxx_t| passed to it a dummy argument name of \verb|xxx|
  should be used if the \verb|self| dummy argument name is
  inappropriate.
\end{enumerate}
Appendix~\ref{sec:appa} contains a template object-oriented design
pattern \texttt{oo\_m.f90} for use in \coden.

\section{Avoid using double or trailing underscores}

Don't use names like \verb|my__variable| or \verb|my_variable_|.
Double underscores are hard to distinguish from single underscores
in some fonts.

\section{Use implicit none and private}

Every module should have \verb|implicit none| and \verb|private|
statements. Program units should have \verb|implicit none|.
The \verb|private| attribute can be used in user-defined types
to hide the components of a type.

\section{One statement per line}

Generally, do not put more than one statement per line (don't use the \verb|;|
statement separator). However, this practice may be helpful if the statements are
simple equalities and involve closely related variables, e.g.\ assignments to different components
of a vector or derived type.

% {\it
% Using more than one statement per line:
% \begin{enumerate}
% \item makes the code harder to read,
% \item makes it harder to set break points in debuggers,
% \item makes it harder to analyse the output of profiling
%    and coverage tools,
% \item makes it harder to associate compiler error messages with
%    statements and
% \item confuses some utilities (a2ps pretty printer).
% \end{enumerate}
% }

\section{Indenting}

The source code should use indenting.
The indenting style is as follows. Statements that logically subdivide
the source (such as \T{module}, \T{subroutine}, \T{end function}) begin in column~$1$.
All other statements are indented at least~$2$ characters, except statements
with numeric labels.

The extra indentation used in blocks such as \T{if}, \T{do}, should be
$2$ more characters per level of indentation.
%\begin{tabular}[h]{ll}
%do blocks: &  3 characters\\
%if/select case/where/forall blocks: & 3 characters \\
%type/interface blocks: &  3 characters \\
%program/module/subroutine/function blocks: & 2 characters
%\end{tabular}\\[2ex]
There should be no extra indentation applied to continuation lines, which
should be denoted by a trailing ampersand \verb|&|. Optionally \verb|&|
may appear in column~$2$.
Comments should be indented so that the initial \verb|!| lines up with
what they are commenting on.
%
%The extra indentation used in different kinds of blocks should be:\\[2ex]
%\begin{tabular}[h]{ll}
%do blocks: &  3 characters\\
%if/select case/where/forall blocks: & 3 characters \\
%type/interface blocks: &  3 characters \\
%program/module/subroutine/function blocks: & 2 characters
%\end{tabular}\\[2ex]
%The extra indentation applied to continuation lines should be 5
%characters. (The Xemacs default.)\\
%Comments should be indented so that the initial \verb|!| lines up with
%what they are commenting on.


\section{Continuation lines}

Use an ampersand, \verb|&|, at the end of the line to be continued
and in column~$2$ of the continuation line if desired.
%and at the start of the continuation line. The second and subsequent
%\verb|&| should be indented an extra 5 characters.

\section{Labelling}

\begin{itemize}
\item Numeric labels should not be used except in the case of the
  number 9999 or 999 and then only for an error/emergency exit case.
\item The \verb|goto| statement can be used to break out of a loop but 
  \verb|exit| and \verb|cycle| are preferred where possible.
\item Longer \verb|do| loops should be labelled with a meaningful text 
  identifier to aid the understanding of programmers who are not
  familiar with the code.
\item \verb|do| loops with \verb|exit| statements are preferred to the 
  \verb|do while| construct which is constrained to test the logical
  clause at the beginning of the loop whereas an \verb|exit| statement 
  can occur at any point in a loop.
\end{itemize}

\section{Format statements}
In order of preference:
\begin{enumerate}
\item From a character variable,
\item embedded in the \verb|write| statement, e.g.\
\verb|write(luout,'(i2,a)')...|,
\item Numbered \verb|format| statements.
\end{enumerate}

\section{Keywords and spaces}

The preferred style is to use spaces in \verb|else if|, \verb|end do|,
\verb|end subroutine| etc. 

\verb|goto| is preferred to \verb|go to|.

\section{Declarations}

All parts of the declaration of a variable should be in a single
statement rather than having multiple statements with the declaration 
scattered:
\begin{verbatim}
  real(kr2), dimension(:,:), target, allocatable, save, public :: x
\end{verbatim}
is preferred over:
\begin{verbatim}
  real(kr2) x
  ... 
  dimension x(:,:)
  ...
  allocatable x
  ...
  target    x
  ...
  public    x
  ...
  save      x
\end{verbatim}


N.B.\ the form above is preferred over:
\begin{verbatim}
  real(kr2), target, allocatable, save, public :: x(:,:)
\end{verbatim}
because it allows the variable to be initialised in the declaration:
\begin{verbatim}
  real(kr2), public, dimension(3) :: x0 = 0.0_kr2
\end{verbatim}

Similarly, the attributes \verb|intent(...)|, \verb|optional|,
\verb|parameter|, \verb|pointer| and \verb|private| should be used in
type declaration statements in preference to the corresponding
separate declarations.

The new form of character declaration should be used:
\begin{verbatim}
  character(4), parameter :: extension = '.dat'
\end{verbatim}
or
\begin{verbatim}
  character(len=4), parameter :: extension = '.dat'
\end{verbatim}

This is more regular than the older way of specifying the length
(which can only be used with characters with default kind):
\begin{verbatim}
  character*4, parameter :: extension = '.dat'
\end{verbatim}

\section{Comments and documentation}

In \coden, documentation is part of the code. We
propose a particular commenting style which can be processed by
doxygen~\cite{He07doxy} to produce html and/or pdflatex documentation automatically.
doxygen operation is by default controlled by a file named ``doxyfile",
supplied with the source. Note that doxygen~1.8.5  or later is needed
for best results.

The following features of \T{doxyfile} are noted, viz.
the setting of the tag\\
%\begin{verbatim}
\verb|PROJECT_NAME = | \coden\  \\
%\end{verbatim}
The html aliases
\begin{verbatim}
ALIASES = Bold{1}="<b>\1</b>"
ALIASES += Emph{1}="<em>\1</em>"
ALIASES += Tt{1}="<tt>\1</tt>"
\end{verbatim}
%and the words removed to produce abbreviations
%\begin{verbatim}
%ABBREVIATE_BRIEF       = "The $name class" \
%                         "The $name widget" \
%                         "The $name file" \
%                         is \ 
%                         provides \
%                         specifies \
%                         contains \
%                         represents \
%                         a \  
%                         an \ 
%                         the  
%\end{verbatim}

doxygen uses \verb|!>| to indicate that the comment refers to the following
Fortran source, \verb|!<| to the preceding Fortran source and \verb|!!| (double
exclamation marks) to extend a comment over a line.  The comments
applying to a module, subroutine, function or variable should be
placed immediately before the item they are describing, ie.\ use \verb|!>|.
Comments can also be placed at the end of a line (using \verb|<!|) but
because of space limitations
this is usually only suitable for declarations of variables which
don't require much description.  With the \T{ALIASES} tag set as above
text enclosed by braces following \verb|\Emph|, \verb|\Bold|  and \verb|\Tt|  is set
respectively to the emphasised, strong and monotype fonts.
Indented text starting with `-' is set as itemised lists in the
output from doxygen.  Alternatively, html markup can be included
in the comments in the code, but this makes it less readable in the
code listing. Mathematical quantities can be set in LaTex by enclosing the LaTeX
formulae by the strings \verb|\f$| and \verb|\f$|.

Appendix~\ref{sec:appa} shows listings of sample module files: 
\begin{enumerate}
\item a template file \texttt{template\_m.f90} showing the standard
  usage of the doxygen formatted comments.
\item an object-oriented pattern file \texttt{oo\_m.f90} showing doxygen
  comments in the OO factory pattern to be used in \coden\ and
\item a second, bigger  file \texttt{bigtobj\_m.f90} showing the pattern
  to be followed for objects with associated file handling.
\item a object definition file \texttt{bigobj\_h.f90} to go with
  objects with associated file handling, or which have many different uses.
\item a program control object-oriented methods module
  \texttt{qcontrol\_m.f90} for use with program \texttt{qprog}. This reads
  in data from separate namelists which defines files for the program (as
  distinct from its modules), a set of parameters for the program and a
  set of output control options for the program.
\item a program control object definition module
  \texttt{qcontrol\_h.f90} for use with module \texttt{qcontrol\_m}.
\item Main program \texttt{qprog.f90} for program \texttt{qprog}.

\end{enumerate}
%Appendix ~\ref{sec:appb}, actually now a separate document, shows the
%doxygen output for these sample modules.

Documentation of the Fortran 95 program files shall take the following
form:
\begin{enumerate}
\item Each file shall have the rubric as
  specified in Section~\ref{sec:prot-mark-git} and shown in the
  examples in Appendix~\ref{sec:appa}.
\item Every module should have after the rubric comments giving the
  summary detailed design documentation associated with it. This falls
  under the following five headings, the last two of which are optional
  if doxygen has been used to produce html documentation:
  \begin{description}
  \item[Type]
 The type should be defined by stating its logical and physical
 characteristics.

 The \emph{logical characteristics} should be stated by saying which of the
 following categories the module belongs to (more than one may apply and
 others may be added):
 \begin{itemize}
    \item control data input and checking
    \item file handling
    \item geometrical processing
    \item framework and infrastructure
    \item magnetic field processing
 \end{itemize}

 The \emph{physical characteristics} should be defined by stating which of
 the following descriptions apply to the module (more than one
 may apply and others may be added):
 \begin{itemize}
   \item data module
   \item execution module
   \item function
   \item main program unit
   \item object-oriented module
   \item subroutine
 \end{itemize}
   \item[Purpose]
 This section describes the purpose of the module.
 It traces back to the software requirements which justify the existence 
 of the module either explicitly by citing the software 
 requirement or implicitly by saying that this component is a 
 subordinate of another module.
    \item[Function]
 This section describes the function of the module.
 It should describe what is does rather than how it does it,
 which should be described in the processing section.
  \item[Processing] (Optional)
 This section describes the processing done by the module.
 It should describe how the module does what it does.
 The use of pseudocode is encouraged.
  \item[Dependencies]  (Optional)
 This section describes what needs to have been done before this
 module and its subordinate modules can be used.
  \end{description}
\item Every subroutine or function should have a comment associated
  with it saying what it does. A single line comment is usually
  inadequate, but may be sufficient if details of the code are published
  or described elsewhere and doxygen-produced documentation is available.
\item Every variable and type should have a comment associated with it
  saying what it is. Public variables and dummy arguments will
  normally need longer descriptions than local variables. Every
  component of a derived type should have a comment. Multidimensional
  arrays will normally need longer descriptions than scalar variables
  because it is necessary to describe what each dimension represents.
\item Comments for all public quantities should begin with \verb|!>| or \verb|!<|
  so that doxygen picks them up, with \verb|!!| used for intermediate
  comment lines. Private quantities can be commented using
  \verb|!| (or isolated \verb|!!|) depending on whether or not the programmer deems
  it helpful for the information to be included in the html detailed
  design documentation generated by doxygen.
\end{enumerate}

The use of horizontal ruled lines to separate subprograms in modules
should be used to aid readability of the program listing.


\section{Protective markings and version control}\label{sec:prot-mark-git}

Every source file should contain the following header at the top
of the file.  The text between the \verb|!begin rubric| and 
\verb|!end rubric| can be updated automatically when necessary.
%{\small
%\begin{verbatim}
%!begin rubric
%!******************************************************************************
%!*                United Kingdom Atomic Energy Authority                      *
%!*                                                                            *
%!*                                                                            *
%!*                       SMITER 2015 RELEASE 1.0                              *
%!*                                                                            *
%!******************************************************************************
%!*                                                                            *
%!* This is an unpublished work created in 2015, any copyright in which vests  *
%!* in UKAEA and ITER Organisation. All rights reserved.                       *
%!*                                                                            *
%!* The information contained in this software is proprietary to UKAEA         *
%!* and ITER Organisation unless stated otherwise and is                       *
%!* made available in confidence; it must not be used or disclosed without the *
%!* express written consent of UKAEA. This document may not be copied in whole *
%!* or in part in any form without the express written consent of UKAEA.       *
%!*                                                                            *
%!* Author: Wayne Arter                                                        *
%! Copyright (c) 2015, UKAEA and ITER Organisation                             *
%!******************************************************************************
%!
%!--------------------------------------------------------------------------
%!end rubric
%\end{verbatim}
%}
{\small
\begin{verbatim}
!begin rubric
!******************************************************************************
!*                United Kingdom Atomic Energy Authority                      *
!*                                                                            *
!*                                                                            *
!*                       SMITER 2015 RELEASE 1.0                              *
!*                                                                            *
!******************************************************************************
!*                                                                            *
!* This is an unpublished work created in 2015. It may be copied,             *
!* distributed and used as set out in the standard terms and conditions for   *
!* use of IMAS software.                                                      * 
!*                                                                            *
!* Since these conditions had not been finalised at time of delivery, there   *
!* is the rider that IMAS terms and conditions are expected to exclude the    *
!* use of the software for activities unrelated to the ITER project.          *
!* As an interim measure, it is here explicitly stated that this software     *
!* must not be used for for nuclear reactor design activities unrelated to    *
!* ITER  without the express written consent of UKAEA.                        *
!*                                                                            *
!* Author: Wayne Arter                                                        *
! Copyright (c) 2015, UKAEA and ITER Organisation                             *
!******************************************************************************
!
!--------------------------------------------------------------------------
!end rubric
\end{verbatim}
}


Version control will be by \T{git}. Different deliveries to the customer
should have an explicit (different) Release number in the rubric.
For development work, the branch of the git repository used by the developer
needs to be recorded, then the developer must ensure that the code executed
is the latest version on the branch \emph{or} explicitly record the version.

\section{File templates}
Template files are shown in Appendix~\ref{sec:appa}. The first one
\verb+template_m.f90+ shows the general layout and documentation for a
module, and the second one, \verb+oo_m.f90+, gives the design pattern
for an object-oriented module. Developers should copy the appropriate
template when creating a new module.  %Further template files may be added later.

Template files should be included in the git repository.
Editors such as Xemacs should be customised so that the template file
is available from a menu to encourage developers to make use of the
template.

\section{Extensions}

The only extensions to Fortran 95 that should be used are the use of
\verb|getarg| and \verb|iargc| for accessing the command line
arguments and the use of \verb|flush| for flushing output to files.
 Fortran 2003 introduces standard 
routines \verb|command_argument_count| and \verb|get_command_argument|
but currently these are not as widely available as \verb|getarg| and
\verb|iargc|. \verb|flush| is widely available and is part of Fortran
2003.

It is assumed that the Fortran 95 compiler used is compliant with
extensions defined in ISO/IEC TR 15581:2001(E)~\cite{tr15581}, so that
allocatable arrays are allowed as components of derived types. Note
that this ISO document is formally a revision of the Fortran 95
standard.

There are many other features of Fortran 2003 which are
desirable but none is as vital as the features mentioned above and
they should be avoided.



\section{Specify intent for dummy arguments}

Developers should always specify the intent for the dummy arguments of
subroutines and functions because doing so can help detect errors. It
also makes the code easier to understand and may allow the compiler to
produce faster code.


\section{Use pure or elemental}
Use the \verb|pure| or \verb|elemental| prefix for procedures which
are pure or elemental. This can make them easier to understand 
and allows them to be used in ways which may enhance the performance of
the code.

\section{Use only}

Use the \verb|only| keyword with \verb|use| statements if only a few
names from a module are being used. This makes the code easier to
understand and more robust, however is arguably the least important of
all the requirements placed on the developer.

\section{Use keywords for long argument lists}

Use keywords for argument lists if there are more than a few
arguments. The exact point at which keywords should start to be used
depends on how regular the argument list is. Consider using
optional arguments if some of the arguments can have default values
or are not always needed.

\section{Large automatic arrays}

Do not use large automatic arrays. Allocate storage
for large arrays using the \verb|allocate| statement.
Automatic arrays are usually put on the stack and stack size is
usually limited.

\section{Target attribute}

Use the \verb|target| attribute with the \verb|allocatable| attribute
when you are going to allocate storage and associate a pointer with
it. This makes the design intention clearer than using a \verb|pointer|
attribute and may allow the compiler to produce faster code (although
there should be no significant speed difference with a good compiler).
The \verb|pointer| attribute should be reserved for when you are using
it to define an alias.

It is assumed that the Fortran 95 compiler used is compliant with
extensions defined in ISO/IEC TR 15581:2001(E)~\cite{tr15581}, so that
allocatable arrays are allowed as components of derived types. If it
is not, then the \verb|pointer| attribute must be used instead to
define allocatable components of a derived type.

Do not use
{\it 
\begin{verbatim}
  integer, pointer :: grpcode(:)
  type(node_tree), pointer, public :: treenode(:)
  ...
  allocate(grpcode(ngrp), stat=istat)
  ...
  allocate(treenode(nnode), stat=istat)
\end{verbatim}
when the following would be clearer:
\begin{verbatim}
  integer, allocatable, target :: grpcode(:)
  type(node_tree), allocatable, target, public :: treenode(:)
  ...
  allocate(grpcode(ngrp), stat=istat)
  ...
  allocate(treenode(nnode), stat=istat)
\end{verbatim}
}

\section{Deprecated features}

Avoid \texttt{equiv\-alence} statements, \verb|common| blocks,
the \verb|block data| program unit, and the \verb|entry| statement.

\section{Shape disagreement}

Avoid shape disagreement between dummy arguments and actual arguments 
unless absolutely necessary. This type of disagreement will usually
only be necessary if it is wished to view the array as having different
ranks in different parts of the code. \verb|reshape| could be used
to avoid the disagreement but this involves two additional copy
operations. Code which uses shape disagreement should make the 
implicit reshaping of the array as explicit as possible:

\begin{verbatim}
real, dimension(1000) :: x

call comp( y=x(101:200), & ! Effectively y=reshape( x(101:200), (/10,10/) ) 
   &  m=10 , n=10 )
...
subroutine comp(y, m, n)
real, dimension(m,n), intent(inout) :: y
...
\end{verbatim}


\section{Character length disagreement}

Avoid  character length disagreement between dummy arguments
and actual arguments.

\section{Redundant features}

Avoid redundant features. These include the \verb|include| line, the
\verb|do while| loop construct, the \verb|dimension| and
\verb|parameter| statements and specific names of intrinsic functions
e.g. \verb|float|, \verb|dsin| etc.

Use \verb|do| and \verb|end do| instead of using statements with
numerical labels. 

Use the new forms of relation operators \verb|==|, \verb|/=|,
\verb|<|, \verb|<=|, \verb|>|, and \verb|>=| instead of the old
forms. {\it The new forms are more compact and easier to read than the 
  old forms}.

\section{Use simple constructs}

Use simple constructs. Complicated or obscure constructs
may be hard to understand and are more likely to suffer from compiler
bugs. 

Do not replace simple code with more complicated
code to avoid bugs in particular compilers.
Get the compilers fixed or if absolutely necessary issue temporary
patches for the code for use with the buggy compilers.


The developer should not optimise code prematurely. If it is found
necessary to optimise code in a way which makes it harder to
understand the developer should consider having a switch which
allows the simpler version to be selected.

\section{Include stubs for future development}

Include comments, derived types, and dummy routines for future
extensions. These can act as a reminder for when the features need 
to be implemented. Make sure unused features are marked ``not used''
or are commented out with a clear comment about what they are for.

\section{Named constants}

Consider using named constants rather than literal values to make
changing the types of variables easier:
\begin{verbatim}
  real(rk2), parameter :: one = 1.0_kr2
\end{verbatim}
The module \verb|const_numphys_h| provides numerical values for a small
range of different named numeric constants, such as~$\pi$.
Inspect the contents of the module to see what is currently available,
and consider adding new numeric constants here, observing the
convention that all such variables begin \verb|const_|.

\chapter{Architectural Recommendations}
\label{sec:arch}

This section describes recommendations which are architectural in nature
although some of the recommendations in Section~\ref{sec:style} may be
considered as being architectural as well.


\section{Explicit interfaces}

This is the most important architectural recommendation. 
Explicit interfaces must be used throughout the code.
This means using Fortran 95 modules (or possibly interface blocks
to interface to legacy code). The use of explicit interfaces
allows the compiler to check that actual and dummy arguments
match in number and type, which reduces programming errors.
Explicit interfaces also allow the use of keywords, optional arguments
and generic interfaces which make the code easier to understand and
maintain.

\section{Parameterised types}

Do not hardwire the types of real and integer variables.
It may be helpful both for code development and distribution
on different hardware to be able to change easily between 4, 8
and 16~byte real numbers for
different variables, and between 1, 2, 4, and 8~byte integers for
different variables (codes routinely overflow 4~byte
integers in computing how much memory or disk space is needed).
The module \T{const\_kind\_m} is provided for this purpose,
and should be inspected to ascertain the available types.

\section{Save attributes}

Module variables may become undefined if the module goes out of
scope. Either use the save attribute on module variables or ensure that
the module remains in scope. It is safer to use the \verb|save|
attribute on module variables because it is difficult to make sure
that the module variables remain in scope.


\section{Check status flags}

Always check the error status from \verb|open|,
\verb|read|, \verb|write| and \verb|allocate|.
The module \T{log\_m} is provided for this purpose, and contains
subroutines
\begin{verbatim}
       log_open_check ---- checks file opening
       log_read_check ---- checks result of read
       log_write_check ---- checks result of write
       log_alloc_check ---- checks array allocation
\end{verbatim}
The errors from \verb|close|, \verb|deallocate| may also be
significant, but can largely be avoided for example by the use of 
\begin{verbatim}
        if (allocated(array)) deallocate(array)
\end{verbatim}


If it is necessary to deal with status flags in line, the style:
\begin{verbatim}
  write(luout,stat=istat)
  if ( istat /= 0 ) then
    ...
  end if
  ...
  write(luout,stat=istat)
  if ( istat /= 0 ) then
    ...
  end if
\end{verbatim}
is better than
\begin{verbatim}
  write(luout,err=999)
  ...
  write(luout,err=999)
  ...
  999 continue
  ...
\end{verbatim}
because the error handling code is closer to the place where the error
occurs. 



\section{Choice of data types}

Fortran 95 has a much richer choice of data types and design patterns
available than Fortran 77. Do not just use the Fortran 77 approach without
considering the alternatives.

One example where there are more choices is in how to have an array
of variable-sized arrays. There are three main ways to do this:
\begin{enumerate}
\item have an array of indices pointing into a single array
which actually holds the data,
\item have an array of pointers to arrays which are associated
  with a section of a single array or
\item have an array of pointers to arrays which are associated
   with storage using the \verb|allocate| statement.
\end{enumerate}
Fortran 95 does not allow arrays of pointers which are required for
options 2 and 3  but it does allow arrays of derived types which
contain a single component which is a pointer to an array:
\begin{verbatim}
 type array_ptr_t
   real(dbl), dimension(:), pointer :: ptr !! pointer to array
 end type array_ptr_t
 ...
 type(array_ptr_t), dimension(10) :: x !! array of pointers to arrays
 ...
 real(dbl), dimension(100), target : storage
 ...
 x(2)%ptr => storage(5:104)          ! case 2
 allocate (x(1)%ptr(100),stat=istat) ! case 3
\end{verbatim}
Option 3 is the safest option  and option 1 is the least safe option.
Option 2 is intermediate but may be useful if developers want to
manage storage themselves instead of letting the system manage
storage with \verb|allocate| and \verb|deallocate|.
An implementation using option 3 can easily be changed to use option
2 by replacing the \verb|allocate| statement with
code to manage storage in an array. 

Options 2 and 3 use slightly more 
storage than option 1 because a pointer will usually consist of an
address and a small number of integers giving the rank,
dimensions and strides of the array. The pointer probably 
needs between 3 and 6 times as much storage as a simple integer index.
This is usually a small price to pay for the extra safety and improved 
clarity of the code.

With option 3 it would be better to have an array of allocatable
arrays.  This can be realised by using a derived type with a single
component which is an allocatable array, instead of a pointer as
described above. Note that this requires the
TR15581~\cite{tr15581} extension to Fortran 95. We shall assume that
compilers for \coden\ will have this extension.
% {\it If allocatable arrays were used the compiler would know that the 
%   arrays were contiguous and did not have non-unit strides or vector
%   subscripts. Using allocatable arrays would also make the intention of
%   developers clearer i.e.\ that they are intending to allocate
%   storage rather than point to existing storage.}



\section{Record lengths for unformatted direct access files}

The specification of record length for unformatted direct access input 
and output is mandatory but the unit used in the \verb|recl=|
specifier in an \verb|open| statement is unspecified.

The \verb|iolength=| specifier should be used with the \verb|inquire|
statement to determine the record length which should be used.
The \verb|inquire| statement should be used with all possible
input/output lists and the maximum length used. This avoids
making assumptions about how data are packed:
\begin{verbatim}
real(kr2), dimension(10) :: x
integer(ki3), dimension(5) :: i
integer :: length_max, length
...
length_max = 0
inquire(iolength=length) x(1:5),i(1:2)
length_max = max(length, length_max)
inquire(iolength=length) x(1:4),i(1:3)
length_max = max(length, length_max)
...
open(..., recl=length_max, ...)
...
write(...)  x(1:5),i(1:2)
write(...)  x(1:4),i(1:3)
\end{verbatim}
It is however reasonable to assume that inserting an extra item in an
input/output list will not make the record length shorter so that in
the example above it would also be acceptable to have:
\begin{verbatim}
write(...)  x(1:5),i(1)
write(...)  x(1:4),i(1:2)
write(...)  x(1:4),i(1:3)
...
\end{verbatim}

We don't expect to use formatted direct access files.

\section{Estimating storage sizes}
\label{sec:sizeof}

The amount of storage required for a type can be estimated using
the \verb|size| and \verb|transfer| functions:

\begin{verbatim}
program sizeof
implicit none
integer, parameter :: ki1 = selected_int_kind(2)
integer, parameter :: ki1_bits = bit_size(1_ki1)
integer, parameter :: ki1_bytes = ki1_bits/8
type x 
  real :: y
  character :: c
  integer :: i
end type x
integer, parameter :: real_bytes = ki1_bytes*size(transfer(1.0,(/1_ki1/)))
integer, parameter :: char_bytes = ki1_bytes*size(transfer(' ',(/1_ki1/)))
integer, parameter :: int_bytes = ki1_bytes*size(transfer(1,(/1_ki1/)))
integer, parameter :: x_bytes = ki1_bytes*size(transfer(x(1.0,'',1),(/1_ki1/)))
write(*,*) 'integer kind ',ki1,' contains ',ki1_bits,  ' bits'
write(*,*) 'integer kind ',ki1,' contains ',ki1_bytes,  ' bytes'
write(*,*) 'real contains ',real_bytes,' bytes'
write(*,*) 'character contains ',char_bytes,' bytes'
write(*,*) 'integer contains ',int_bytes,' bytes'
write(*,*) 'type x contains ',x_bytes,' bytes'
end program sizeof
\end{verbatim}


% \section{Maximum record lengths for sequential access files}

% {\it Should we use \verb|recl=| specifier to specify maximum record
%   lengths in \verb|open| statements for sequential access files with
%   unusually long line lengths or rely on errors being detected
%   in the i/o statements?} 

  
\section{Precision and kinds}
The module \verb|const_kind_m| provides  a set of commonly used variable
precisions, such as are eg.\ equivalent to single and double precision reals.
Inspect the contents of the module to see what is currently available,
and consider adding new variable kinds here, observing the
convention that all such variables begin \verb|k|.
%\section{Precision and kinds}
%
%We plan to have a precision module which defines the kinds of 
%real, complex and integer variables that are used in the software.
%A possible sketch implementation would be:
%
%{\small
%\begin{verbatim}
%module precision_m
%   implicit none
%! set parameters using selected_real_kind and selected_int_kind
%   integer, parameter, public :: precision_kr1 = ... !! kind for 4 byte reals
%   integer, parameter, public :: precision_kr2 = ... !! kind for 8 byte reals
%   integer, parameter, public :: precision_ki1 = ... !! kind for 1 byte integers
%   integer, parameter, public :: precision_ki2 = ... !! kind for 2 byte integers
%   integer, parameter, public :: precision_ki3 = ... !! kind for 4 byte integers
%   integer, parameter, public :: precision_ki4 = ... !! kind for 8 byte integers
%end module precision_m
%\end{verbatim}
%}
%
%There are three issues with this design:
%\begin{enumerate}
%\item The \verb|precision_kr1| could be the same as
%  \verb|precision_kr2| (and similarly for other types). This
%   could be a problem if generic interfaces are used because
%   there might then be two versions of the same specific implementation.
% \item The kind parameters above may have negative values if the
%   appropriate kind is not available. This should be detected at
%   compile time.
% \item Fortran uses default integer kind for many purposes.
%\end{enumerate}
%
\section{Error handler and program termination}

Routines should not use the \verb|stop| statement directly.
The program should only be terminated using a standard interface,
via a routine defined in the \verb|log_m| module,
specifically \verb|log_error|. 
The complete error handling and logging system also requires
routines \verb|log_init| and \verb|log_close|, and there is
a facility to record information using subroutine \verb|log_value|.

%\section{Extensible user input}
%
%There should be a way of extending the control file input
%to allow developers to add flags and options for development
%purposes. There should be a module which contains a 
%registry of these options.
%
%One possible design is to have a module called \verb|expert|
%which holds the registry of options and utility routines
%for accessing them.
%
%Possible applications of the options handled by the \verb|expert|
%module could be:
%\begin{enumerate}
%\item Choose between BLAS routines and standard Fortran code,
%\item choose between different implementations,
%\item turn on extra diagnostics and checks,
%\item activate self tests,
%\item request version inventory,
%\item change edit descriptors.
%\end{enumerate}
%
%The \verb|expert| module can be used for many things that the C conditional
%compilation directive \verb|#ifdef| is used for.
%
%Useful items can be promoted to standard input tags.
%
%We would want most modules to be able to use the \verb|expert| module
%which means that the \verb|expert| module will be able to use very few
%of the other modules.
%
%\section{Self tests}
%
%Consider using the \verb|expert| module to activate self tests.

\section{Code monitoring}
The module \verb|clock_m| is provided to measure cpu times between different
calls. Inspect the contents of the module to see what is currently available.

The module \verb|date_time_m| provides functions which return the current
date in either a long or short format, and its routines should be used
whenever dates or times are required.

\section{Physical and numerical constants}
The module \verb|const_numphys_h| provides numerical values for a small
range of different physical and numerical constants, including $\pi$.
Inspect the contents of the module to see what is currently available,
and consider adding new physical constants here, observing the
convention that all such variables begin \verb|const_|.

\section{Selection of I/O units}

A module should be provided for allocating I/O units. This enhances
portability by avoiding assumptions about which unit numbers are used
for standard purposes.



\clearpage

%---------------------------------------------------------------------
%
%The references
%--------------

%using bibTeX

\bibliographystyle{unsrt}
\bibliography{new,waynes,misc,warv}

%-----------------------------------------------------------------
%             example of how to add appendix
%
% if you want to use the following, remove the above \end{document} 

\cleardoublepage
\appendix
\addcontentsline{toc}{part}{APPENDICES}

\part*{APPENDICES}
\chapter{Sample Modules}\label{sec:appa}
%\addtocounter{page}{10}

\section{A template module}
\label{sec:template}
\setlength{\oddsidemargin}{0.0mm}
\setlength{\evensidemargin}{\oddsidemargin}

\small
\begin{verbatim}
module template_m

  use another_m, only: another_var1 !<  Use only another_var1 from this module
  use second_m,  second_var1 => v   !<  Use all module with second_var renamed v
  use precision_m, only krb         !<  Use base real precision

  implicit none
  private

! public subroutines

  public template_usersub

! public types

!> Example type 1
  type,public :: template_xyz_t
     integer :: num = 0                !< Number of things
     integer,pointer :: ptr => null() !< Pointer to first thing
  end type template_xyz_t

! public variables

!> The description of the first public variable in the module.
!! Public variables will usually require at least a full line of
!! description.
!!
  integer,dimension(1:3),public :: template_eg1 !< local variable
!> second public variable
  integer,dimension(:),allocatable,public :: template_eg2 !< local variable

! private variables

  integer,dimension(1:3) :: eg1 !< example variable 1
!> Example variable 2 has a longer description
!! which is too long to fit at the end of the line.
  integer,dimension(1:3) :: eg2 = 0 !< local variable
  integer,dimension(1:3) :: eg3 = 0            !< example variable 3
  integer,dimension(:),allocatable :: eg4     !< example variable 4
  integer,target                    :: eg5 = 0 !< example variable 5


  contains
!--------------------------------------------------------------------------
!> First user subroutine with at least a line of description
!! saying what it does.
!!
subroutine template_usersub(arg1)
  ! arguments
  real(krb), intent(in) :: arg1 !< argument 1
  ! local variables
  integer :: i !< local variable description
  integer :: j !< local variable description
  integer :: k !< local variable description

  ! user code
  call module_xyz()

end subroutine template_usersub
!--------------------------------------------------------------------------
!> Second user subroutine with at least a line of description
!! saying what it does.
!!
subroutine template_usersub2(arg2)
  ! arguments
  integer(krb), intent(out) :: arg2 !< argument 1
  ! local variables
  integer :: i !< local variable description
  integer :: j !< local variable description
  integer :: k !< local variable description

  ! user code
  arg2=10

end subroutine template_usersub
!--------------------------------------------------------------------------

end module template_m
\end{verbatim}
\normalsize

%% To generate files for including here, start with the file.f90 and
%% (1)a2ps -R -B --highlight-level=heavy --columns=1 -o file.ps file.f90
%% (2) split file.ps into 1 page files using
%%    psselect -p1 file.ps file.p1.ps
%% etcetera
%% (3) convert files to eps
%%   ps2eps file.p1.ps
%% etcetera
%% (4) if pdf is wanted use epstopdf
%\enlargethispage*{30pt} 
%\begin{center}
%    \includegraphics[width=0.9\textwidth]{template_m_p1}
%\end{center}
%\clearpage
%\begin{center}
%  \includegraphics[width=0.99\textwidth]{template_m_p2}
%\end{center}
%\clearpage
%\begin{center}
%  \includegraphics[width=0.99\textwidth]{template_m_p3}
%\end{center}


\section{Object-oriented factory pattern module}
\label{sec:oopattern}

\small
\begin{verbatim}
module oo_m

! use statements

! default to explicit typing and private
  implicit none
  private

! public subroutines

  public oo_init
  public oo_delete
  public oo_method1

! public types

!> Type to contain non-static components of objects.
!! All components should have default initial values if possible.
  type, public :: oo_t
     integer :: comp1 = 0  !< component 1
     integer :: comp2 = 0  !< component 2
  end type oo_t

! public variables

! private types

! private variables

  contains
!--------------------------------------------------------------------------
!> Constructor for object.
subroutine oo_init(self)

  ! arguments
  !> Variable to contain nonstatic data for self object must be first argument and be called \Tt{self}
  type(oo_t), intent(out) :: self !< local variable
end subroutine oo_init
!--------------------------------------------------------------------------
!> First method for module.
subroutine oo_method1(self,other_oo)

  ! arguments
  type(oo_t), intent(inout) :: self !< Data for self object must be first argument and be called \Tt{self}
  type(oo_t), intent(in) :: other_oo !< Other object of same type as self

end subroutine oo_method1
!--------------------------------------------------------------------------
!> Destructor for object.
!!
!! The destructor deallocates all storage which the object is
!! responsible for. It is good programming style to make sure that the
!! object cannot be used after it has been deleted by invalidating
!! all the data i.e. nullifying pointers and setting non-pointer
!! variables to values which will generate an error if used.
subroutine oo_delete(self)
  ! arguments
  type(oo_t), intent(inout) :: self !< data for self object
end subroutine oo_delete
!--------------------------------------------------------------------------

end module oo_m
\end{verbatim}
\normalsize

\clearpage


\section{Big object-oriented methods module}
\label{sec:oomethods}

\small
\begin{verbatim}
module bigobj_m

  use bigobj_h
  use log_m
  use const_numphys_h
  use const_kind_m

  implicit none
  private

! public subroutines
  public :: &
  bigobj_initfile,  & !< open file
  bigobj_readcon,  & !< read data from file
  bigobj_solve,  & !< generic subroutine
  bigobj_userdefined,  & !< user-defined function
  bigobj_fn, &  !< general external function call
  bigobj_dia, &  !< object diagnostics to log file
  bigobj_initwrite, & !< open new file, making up name
  bigobj_write, &  !< write out object
  bigobj_writeg, &  !< write out object as gnuplot
  bigobj_writev, &  !< write out object as vtk
  bigobj_delete, & !< delete object
  bigobj_close, & !< close file
  bigobj_closewrite !< close write file

! private variables
  character(*), parameter :: m_name='bigobj_m' !< module name
  integer(ki4)  :: status   !< error status
  integer(ki4), save  :: ninbo=5     !< control file unit number
  integer(ki4), save  :: noutbo=6      !< output file unit number
  character(len=80), save :: controlfile !< control file name
  character(len=80), save :: outputfile !< output file name
  integer(ki4) :: i !< loop counter
  integer(ki4) :: j !< loop counter
  integer(ki4) :: k !< loop counter
  integer(ki4) :: l !< loop counter
  integer(ki4) :: ij !< loop counter
  integer(ki4)  :: ilog      !< for namelist dump after error

  contains
!---------------------------------------------------------------------
!> open file
subroutine bigobj_initfile(file,channel)

  !! arguments
  character(*), intent(in) :: file !< file name
  integer(ki4), intent(out),optional :: channel   !< input channel for object data structure
  !! local
  character(*), parameter :: s_name='bigobj_initfile' !< subroutine name
  logical :: unitused !< flag to test unit is available

  if (trim(file)=='null') then
     call log_error(m_name,s_name,1,log_info,'null filename ignored')
     return
  end if

  !! get file unit
  do i=99,1,-1
     inquire(i,opened=unitused)
     if(.not.unitused)then
        ninbo=i
        if (present(channel)) channel=i
        exit
     end if
  end do

  !! open file
  controlfile=trim(file)
  call log_value("Control data file",trim(controlfile))
  open(unit=ninbo,file=controlfile,status='OLD',iostat=status)
  if(status/=0)then
     !! error opening file
     print '("Fatal error: Unable to open control file, ",a)',controlfile
     call log_error(m_name,s_name,2,error_fatal,'Cannot open control data file')
     stop
  end if

end subroutine bigobj_initfile
!---------------------------------------------------------------------
!> read data from file
subroutine bigobj_readcon(selfn,channel)

  !! arguments
  type(bonumerics_t), intent(out) :: selfn !< type which data will be assigned to
  integer(ki4), intent(in),optional :: channel   !< input channel for object data structure

  !! local
  character(*), parameter :: s_name='bigobj_readcon' !< subroutine name
  character(len=80) :: bigobj_formula !< formula to be used
  integer(ki4), parameter :: MAX_NUMBER_OF_PARAMETERS=10 !< maximum number of parameters allowed
  real(kr8) :: power_split !< variable with meaningful name

  real(kr8), dimension(MAX_NUMBER_OF_PARAMETERS) :: general_real_parameters  !< local variable
  integer(ki4), dimension(MAX_NUMBER_OF_PARAMETERS) :: general_integer_parameters  !< local variable
  integer(ki4) :: number_of_real_parameters  !< local variable
  integer(ki4) :: number_of_integer_parameters  !< local variable

  !! bigobj parameters
  namelist /bigobjparameters/ &
 &power_split, bigobj_formula, &
 &general_real_parameters, number_of_real_parameters, &
 &general_integer_parameters, number_of_integer_parameters

  !! set default bigobj parameters
  power_split=0.5_kr8

  bigobj_formula='unset'
  general_real_parameters=0
  general_integer_parameters=0
  number_of_real_parameters=0
  number_of_integer_parameters=0

  if(present(channel).AND.channel/=0) then
     !! assume unit already open and reading infile
     ninbo=channel
  end if

  !!read bigobj parameters
  read(ninbo,nml=bigobjparameters,iostat=status)
  if(status/=0) then
     !!dump namelist contents to logfile to assist error location
     print '("Fatal error reading bigobj parameters")'
     call log_getunit(ilog)
     write(ilog,nml=bigobjparameters)
     call log_error(m_name,s_name,1,error_fatal,'Error reading bigobj parameters')
  end if

  call lowor(bigobj_formula,1,len_trim(bigobj_formula))
  !! check for valid data

  formula_chosen: select case (bigobj_formula)
  case('unset','exp')
     if(power_split<0.OR.power_split>1) &
 &   call log_error(m_name,s_name,11,error_fatal,'power_split must be >=0 and <=1')

  case('expdouble')
     if(power_split<0.OR.power_split>1) &
 &   call log_error(m_name,s_name,21,error_fatal,'power_split must be >=0 and <=1')

  case('userdefined')
     if(number_of_real_parameters<0) &
 &   call log_error(m_name,s_name,44,error_fatal,'number of real parameters must be >=0')
     if(number_of_real_parameters>MAX_NUMBER_OF_PARAMETERS) then
        call log_value("max number of real parameters",MAX_NUMBER_OF_PARAMETERS)
        call log_error(m_name,s_name,45,error_fatal,'too many parameters: increase MAX_NUMBER_OF_PARAMETERS')
     end if
     if(number_of_integer_parameters<0) &
 &   call log_error(m_name,s_name,46,error_fatal,'number of integer parameters must be >=0')
     if(number_of_integer_parameters>MAX_NUMBER_OF_PARAMETERS) then
        call log_value("max number of integer parameters",MAX_NUMBER_OF_PARAMETERS)
        call log_error(m_name,s_name,47,error_fatal,'too many parameters: increase MAX_NUMBER_OF_PARAMETERS')
     end if
     if(number_of_integer_parameters==0.AND.number_of_real_parameters==0) &
 &   call log_error(m_name,s_name,48,error_fatal,'no parameters set')

  end select formula_chosen

  !! store values
  selfn%formula=bigobj_formula

  selfn%f=power_split

  !! allocate arrays and assign

  selfn%nrpams=number_of_real_parameters
  selfn%nipams=number_of_integer_parameters

  formula_allocate: select case (bigobj_formula)

  case('userdefined')
     if (number_of_real_parameters>0) then
        allocate(selfn%rpar(number_of_real_parameters), stat=status)
        call log_alloc_check(m_name,s_name,65,status)
        selfn%rpar=general_real_parameters(:number_of_real_parameters)
     end if
     if (number_of_integer_parameters>0) then
        allocate(selfn%npar(number_of_integer_parameters), stat=status)
        call log_alloc_check(m_name,s_name,66,status)
        selfn%npar=general_integer_parameters(:number_of_integer_parameters)
     end if
  case default
  end select formula_allocate

end  subroutine bigobj_readcon
!---------------------------------------------------------------------
!> generic subroutine
subroutine bigobj_solve(self)

  !! arguments
  type(bigobj_t), intent(inout) :: self !< module object
  !! local
  character(*), parameter :: s_name='bigobj_solve' !< subroutine name

  self%pow=bigobj_fn(self,0._kr8)

end subroutine bigobj_solve
!---------------------------------------------------------------------
!> output to log file
subroutine bigobj_dia(self)

  !! arguments
  type(bigobj_t), intent(inout) :: self !< module object
  !! local
  character(*), parameter :: s_name='bigobj_dia' !< subroutine name

  call log_value("power ",self%pow)

end subroutine bigobj_dia
!---------------------------------------------------------------------
!> userdefined function
function bigobj_userdefined(self,psi)

  !! arguments
  type(bigobj_t), intent(in) :: self !< module object
  real(kr8) :: bigobj_userdefined !< local variable
  real(kr8), intent(in) :: psi !< position in \f$ \psi \f$

  !! local variables
  character(*), parameter :: s_name='bigobj_userdefined' !< subroutine name
  real(kr8) :: pow !< local variable
  real(kr8) :: zpos !< position
  integer(ki4) :: ilocal !< local integer variable

  zpos=psi
  pow=0._kr8
  !> user defines \Tt{pow} here
  !! .....
  !! return bigobj
  bigobj_userdefined=pow

end function bigobj_userdefined
!---------------------------------------------------------------------
!> general external function call
function bigobj_fn(self,psi)

  !! arguments
  type(bigobj_t), intent(in) :: self !< module object
  real(kr8) :: bigobj_fn !< local variable
  real(kr8), intent(in) :: psi !< position in \f$ \psi \f$

  !! local variables
  character(*), parameter :: s_name='bigobj_fn' !< subroutine name
  real(kr8) :: pow !< local variable

  pow=0._kr8
  !! select bigobj
  formula_chosen: select case (self%n%formula)
  case('userdefined')
     pow=bigobj_userdefined(self,psi)
  end select formula_chosen

  !! return bigobj
  bigobj_fn=pow

end function bigobj_fn
!---------------------------------------------------------------------
!> open new file, making up name
subroutine bigobj_initwrite(fileroot,channel)

  !! arguments
  character(*), intent(in) :: fileroot !< file root
  integer(ki4), intent(out),optional :: channel   !< output channel for object data structure
  !! local
  character(*), parameter :: s_name='bigobj_initwrite' !< subroutine name
  logical :: unitused !< flag to test unit is available
  character(len=80) :: outputfile !< output file name

  !! get file unit
  do i=99,1,-1
     inquire(i,opened=unitused)
     if(.not.unitused)then
        if (present(channel)) channel=i
        exit
     end if
  end do
  noutbo=i

  !! open file
  outputfile=trim(fileroot)//"_bigobj.out"
  call log_value("Control data file",trim(outputfile))
  open(unit=noutbo,file=outputfile,status='NEW',iostat=status)
  if(status/=0)then
     open(unit=noutbo,file=outputfile,status='REPLACE',iostat=status)
  end if
  if(status/=0)then
     !! error opening file
     print '("Fatal error: Unable to open output file, ",a)',outputfile
     call log_error(m_name,s_name,1,error_fatal,'Cannot open output data file')
     stop
  end if

end subroutine bigobj_initwrite
!---------------------------------------------------------------------
!> write bigobj data
subroutine bigobj_write(self,channel)

  !! arguments
  type(bigobj_t), intent(in) :: self   !< bigobj data structure
  integer(ki4), intent(in), optional :: channel   !< output channel for bigobj data structure

  !! local
  character(*), parameter :: s_name='bigobj_write' !< subroutine name
  integer(ki4) :: iout   !< output channel for bigobj data structure

  !! sort out unit
  if(present(channel)) then
     iout=channel
  else
     iout=noutbo
  end if

  write(iout,*,iostat=status) 'bigobj_formula'
  call log_write_check(m_name,s_name,18,status)
  write(iout,*,iostat=status) self%n%formula
  call log_write_check(m_name,s_name,19,status)
  write(iout,*,iostat=status) 'f'
  call log_write_check(m_name,s_name,20,status)
  write(iout,*,iostat=status) self%n%f
  call log_write_check(m_name,s_name,21,status)
  write(iout,*,iostat=status) 'nrpams'
  call log_write_check(m_name,s_name,46,status)
  write(iout,*,iostat=status) self%n%nrpams
  call log_write_check(m_name,s_name,47,status)
  if (self%n%nrpams>0) then
     write(iout,*,iostat=status) 'real_parameters'
     call log_write_check(m_name,s_name,48,status)
     write(iout,*,iostat=status) self%n%rpar
     call log_write_check(m_name,s_name,49,status)
  end if
  write(iout,*,iostat=status) 'nipams'
  call log_write_check(m_name,s_name,50,status)
  write(iout,*,iostat=status) self%n%nipams
  call log_write_check(m_name,s_name,51,status)
  if (self%n%nipams>0) then
     write(iout,*,iostat=status) 'integer_parameters'
     call log_write_check(m_name,s_name,52,status)
     write(iout,*,iostat=status) self%n%npar
     call log_write_check(m_name,s_name,53,status)
  end if

end subroutine bigobj_write
!---------------------------------------------------------------------
!> write object data as gnuplot
subroutine bigobj_writeg(self,select,channel)

  !! arguments
  type(bigobj_t), intent(in) :: self   !< object data structure
  character(*), intent(in) :: select  !< case
  integer(ki4), intent(in), optional :: channel   !< output channel for bigobj data structure

  !! local
  character(*), parameter :: s_name='bigobj_writeg' !< subroutine name
  integer(ki4) :: iout   !< output channel for bigobj data structure

  call log_error(m_name,s_name,1,log_info,'gnuplot file produced')

  plot_type: select case(select)
  case('cartesian')

  case default

  end select plot_type

end subroutine bigobj_writeg
!---------------------------------------------------------------------
!> write object data as vtk
subroutine bigobj_writev(self,select,channel)

  !! arguments
  type(bigobj_t), intent(in) :: self   !< object data structure
  character(*), intent(in) :: select  !< case
  integer(ki4), intent(in), optional :: channel   !< output channel for bigobj data structure

  !! local
  character(*), parameter :: s_name='bigobj_writev' !< subroutine name
  integer(ki4) :: iout   !< output channel for bigobj data structure

  call log_error(m_name,s_name,1,log_info,'vtk file produced')

  plot_type: select case(select)
  case('cartesian')

  case default

  end select plot_type

end subroutine bigobj_writev
!---------------------------------------------------------------------
!> close write file
subroutine bigobj_closewrite

  !! local
  character(*), parameter :: s_name='bigobj_closewrite' !< subroutine name

  !! close file
  close(unit=noutbo,iostat=status)
  if(status/=0)then
     !! error closing file
     print '("Fatal error: Unable to close output file, ",a)',outputfile
     call log_error(m_name,s_name,1,error_fatal,'Cannot close output data file')
     stop
  end if

end subroutine bigobj_closewrite
!---------------------------------------------------------------------
!> delete object
subroutine bigobj_delete(self)

  !! arguments
  type(bigobj_t), intent(inout) :: self !< module object
  !! local
  character(*), parameter :: s_name='bigobj_delete' !< subroutine name

  formula_deallocate: select case (self%n%formula)
  case('userdefined')
     if (self%n%nrpams>0) deallocate(self%n%rpar)
     if (self%n%nipams>0) deallocate(self%n%npar)
  case default
  end select formula_deallocate

end subroutine bigobj_delete
!---------------------------------------------------------------------
!> close file
subroutine bigobj_close

  !! local
  character(*), parameter :: s_name='bigobj_close' !< subroutine name

  !! close file
  close(unit=ninbo,iostat=status)
  if(status/=0)then
     !! error closing file
     print '("Fatal error: Unable to close control file, ",a)',controlfile
     call log_error(m_name,s_name,1,error_fatal,'Cannot close control data file')
     stop
  end if

end subroutine bigobj_close

end module bigobj_m
\end{verbatim}
\normalsize

\clearpage


\section{Big object definition module}
\label{sec:oodefinition}

\small
\begin{verbatim}
module bigobj_h

  use const_kind_m

! public types

! Note these could be merged if the object is instantiated by its parameters  only
!> parameters describing how to construct object
  type, public :: bonumerics_t
     character(len=80) :: formula !< bigobj formula
     real(kr8) :: f !< power split (math variable name allowed)
     integer(ki4) :: nrpams !< number of real parameters
     integer(ki4) :: nipams !< number of integer parameters
     real(kr8), dimension(:), allocatable   :: rpar !< general real parameters
     integer(ki4), dimension(:), allocatable   :: npar !< general integer parameters
  end type bonumerics_t

! type which defines/instantiates the object
  type, public :: bigobj_t
     real(kr8) :: pow !< power
     type(bonumerics_t) :: n !< control  parameters
  end type bigobj_t

end module bigobj_h
\end{verbatim}
\normalsize

\clearpage


\section{Program control object-oriented methods module}
\label{sec:controlmethods}

\small
\begin{verbatim}
module qcontrol_m

  use const_kind_m
  use log_m
  use qcontrol_h
  use bigobj_h
  use bigobj_m

  implicit none
  private

! public subroutines
  public :: &
 &qcontrol_init, & !< open input control data file
 &qcontrol_close, & !< close input control data file
 &qcontrol_read !< read data for this run

! private variables
  character(*), parameter :: m_name='qcontrol_m' !< module name
  integer(ki4)  :: status   !< error status
  integer(ki4), save  :: nin      !< control file unit number
  integer(ki4) :: i !< loop counter
  integer(ki4) :: j !< loop counter
  integer(ki4) :: k !< loop counter
  integer(ki4) :: l !< loop counter
  character(len=80) :: root !< file root

  contains
!---------------------------------------------------------------------
!> open input control data file
subroutine qcontrol_init(fileroot)

  !! arguments
  character(*), intent(in) :: fileroot !< file root
  !! local
  character(*), parameter :: s_name='qcontrol_init' !< subroutine name
  logical :: unitused !< flag to test unit is available
  character(len=80) :: controlfile !< control file name


  !! get file unit
  do i=99,1,-1
     inquire(i,opened=unitused)
     if(.not.unitused)then
        nin=i
        exit
     end if
  end do

  !! open file
  controlfile=trim(fileroot)//".ctl"
  root=fileroot
  call log_value("Control data file",trim(controlfile))
  open(unit=nin,file=controlfile,status='OLD',iostat=status)
  if(status/=0)then
     !! error opening file
     print '("Fatal error: Unable to open control file, ",a)',controlfile
     call log_error(m_name,s_name,1,error_fatal,'Cannot open control data file')
     stop
  end if


end  subroutine qcontrol_init
!---------------------------------------------------------------------
!> close input control data file
subroutine qcontrol_close

  !! local
  character(*), parameter :: s_name='qcontrol_close' !< subroutine name

  !! close file unit
  close(unit=nin)

end  subroutine qcontrol_close
!---------------------------------------------------------------------
!> read data for this run
subroutine qcontrol_read(file,param,bonumerics,plot)

  !! arguments
  type(qfiles_t), intent(out) :: file !< file names
  type(qparams_t), intent(out) :: param !< control parameters
  type(bonumerics_t), intent(out) :: bonumerics !< controls for bigobj
  type(qplots_t), intent(out) :: plot !< plot selectors

  !!local
  character(*), parameter :: s_name='qcontrol_read' !< subroutine name
  logical :: filefound !< true if file exists

  character(len=80) :: prog_input_file  !< qprog data input file
  character(len=80) :: prog_output_file  !< qprog data output file

  character(len=80) :: prog_control  !< option control parameter
  real(kr8) :: prog_realpar !< real control parameter
  integer(ki4) :: prog_intpar !< integer control parameter
  logical :: prog_logicpar !< logical control parameter

  logical :: plot_bigobjout !< bigobj output data selector
  logical :: plot_vtk !< vtk plot selector
  logical :: plot_gnu !< gnuplot plot selector

  !! file names
  namelist /progfiles/ &
 &prog_input_file, &
 &prog_output_file

  !! misc parameters
  namelist /miscparameters/ &
 &prog_control,&
 &prog_realpar, prog_intpar, &
 &prog_logicpar

  !! plot selection parameters
  namelist /plotselections/ &
 &plot_bigobjout, &
 &plot_vtk, &
 &plot_gnu

  !---------------------------------------------------------------------
  !! read input file names
  prog_input_file='null'
  prog_output_file='null'
  read(nin,nml=progfiles,iostat=status)
  if(status/=0) then
     call log_error(m_name,s_name,1,error_fatal,'Error reading input filenames')
     print '("Fatal error reading input filenames")'
  end if

  file%bigobjdata = prog_input_file
  file%out = prog_output_file

  !!check file exists

  call log_value("qprog data file, prog_input_file",trim(file%bigobjdata))
  if(file%bigobjdata.NE.'null') then
     inquire(file=prog_input_file,exist=filefound)
     if(.not.filefound) then
        !! error opening file
        print '("Fatal error: Unable to find qprog data file, ",a)',prog_input_file
        call log_error(m_name,s_name,2,error_fatal,'qprog data file not found')
     end if
  end if

  !! create output file names from root

  !!bigobj file root
  file%bigobjout = trim(root)//"_bigobjout"
  !!vtk file
  file%vtk = trim(root)//"_vtk"
  !!gnu file roots
  file%gnu = trim(root)//"_gnu"

  !---------------------------------------------------------------------
  !! set default misc parameters
  prog_control = 'standard'
  prog_realpar = .0001_kr8
  prog_intpar = 1
  prog_logicpar = .false.

  !!read misc parameters
  read(nin,nml=miscparameters,iostat=status)
  if(status/=0) then
     call log_error(m_name,s_name,10,error_fatal,'Error reading misc parameters')
     print '("Fatal error reading misc parameters")'
  end if

  !! check for valid data
  if(prog_control/='standard') then
     call log_error(m_name,s_name,11,error_fatal,'only standard control allowed')
  end if
  ! positive real parameter
  if(prog_realpar<0._kr8) then
     call log_error(m_name,s_name,20,error_fatal,'realpar must be non-negative')
  end if
  ! positive integer parameter
  if(prog_intpar<=0) then
     call log_error(m_name,s_name,30,error_fatal,'intpar must be positive')
  end if
  if(prog_logicpar) &
 &call log_error(m_name,s_name,40,error_warning,'logicpar is true')

  param%control=prog_control
  param%realpar=prog_realpar
  param%intpar=prog_intpar
  param%logicpar=prog_logicpar

  !---------------------------------------------------------------------
  !! set default plot selections
  plot_bigobjout = .false.
  plot_vtk = .false.
  plot_gnu = .false.

  !!read plot selections
  read(nin,nml=plotselections,iostat=status)
  if(status/=0) then
     call log_error(m_name,s_name,50,error_fatal,'Error reading plot selections')
     print '("Fatal error reading plot selections")'
  end if

  !! store values
  plot%bigobjout = plot_bigobjout
  plot%vtk = plot_vtk
  plot%gnu = plot_gnu

end  subroutine qcontrol_read

end module qcontrol_m
\end{verbatim}
\normalsize

\clearpage


\section{Program control object definition module}
\label{sec:controldefinition}

\small
\begin{verbatim}
module qcontrol_h

  use const_kind_m

!! public types

!> run control parameters
  type, public :: qparams_t
     character(len=80) :: control !< option control parameter
     real(kr8) :: realpar !< real control parameter
     integer(ki4) :: intpar !< integer control parameter
     logical :: logicpar !< logical control parameter
  end type qparams_t

!> file names
  type, public :: qfiles_t
     character(len=80)  :: out       !< output data
     character(len=80)  :: log           !< log file
     character(len=80)  :: bigobjdata         !< bigobj input data file
     character(len=80)  :: bigobjout         !< bigobj output data file
     character(len=80)  :: vtk   !< vtk file
     character(len=80)  :: gnu !< gnuplot file
  end type qfiles_t

!> plot output selectors
  type, public :: qplots_t
     logical  :: bigobjout !< bigobj output data selector
     logical  :: vtk   !< vtk plot selector
     logical  :: gnu !< gnuplot plot selector
  end type qplots_t

end module qcontrol_h
\end{verbatim}
\normalsize

\clearpage


\section{Main program}
\label{sec:mainprogram}

\small
\begin{verbatim}
program p_qprog

  use const_kind_m
  use const_numphys_h
  use date_time_m
  use log_m
  use clock_m
  use gfile_m
  use vfile_m
  use qcontrol_h
  use qcontrol_m
  use bigobj_h
  use bigobj_m

  implicit none

! Local variables
  character(*), parameter :: m_name='qprog' !< module name
  type(qparams_t)     :: param      !< run control parameters
  type(qfiles_t)     :: file      !< names of files
  type(qplots_t)     :: plot      !< diagnostic plot selectors
  type(bigobj_t)  :: bigobj      !< bigobj object
  type(date_time_t) :: timestamp !< timestamp of run
  character(len=80) :: fileroot !< reference name for all files output by run

  integer(ki4):: nin=0 !< unit for other data
  integer(ki4):: nplot=0 !< unit for vtk files
  integer(ki4):: nprint=0 !< unit for gnuplot files
  integer(ki4):: i !< loop variable
  integer(ki4):: j !< loop variable

  character(len=80) :: ibuf !< character workspace
!--------------------------------------------------------------------------
!! initialise timing

  call date_time_init(timestamp)
  call clock_init(40)
  call clock_start(1,'run time')
!--------------------------------------------------------------------------
!! print header

  print *, '----------------------------------------------------'
  print *, 'qprog: one-line program description '
  print *, '----------------------------------------------------'
  print '(a)', timestamp%long
!--------------------------------------------------------------------------
!! file root

!! get file root from arg
  if(command_argument_count()<1) then
!! no file root specified
     print *, 'Fatal error: no file root name specified.'
     print *, 'To run qprog type at the command line:'
     print *, '   qprog fileroot'
     stop
  else
!!get fileroot
     call get_command_argument(1,value=fileroot)
  end if

!! start log
  call log_init(fileroot,timestamp)
!--------------------------------------------------------------------------
!! read control file

  call clock_start(2,'control initialisation time')
  call qcontrol_init(fileroot)
  call qcontrol_read(file,param,bigobj%n,plot)
  call clock_stop(2)
!--------------------------------------------------------------------------
!! other data for bigobj and/or read bigobj controls directly

  call clock_start(15,'initialisation time')
  call bigobj_initfile(file%bigobjdata,nin)
  !call bigobj_readcon(bigobj%n)
  call clock_stop(15)
!--------------------------------------------------------------------------
!! do the main work

  call clock_start(21,'evaluation time')

  select case(param%control)
  case default
     call bigobj_solve(bigobj)
  end select

  call clock_stop(21)
!--------------------------------------------------------------------------
!! data diagnostics to log file - optional

  call clock_start(31,'diagnostics time')

  select case(param%control)
  case default
     call bigobj_dia(bigobj)
  end select

  call clock_stop(31)
!--------------------------------------------------------------------------
!! vtk plot file(s) - optional

  call clock_start(32,'vtk plot time')

  select case(param%control)
  case default
     if(plot%vtk) then
        call vfile_init(file%vtk,'file header',nplot)
        call bigobj_writev(bigobj,'selector',nplot)
        call vfile_close
     end if
  end select

  call clock_stop(32)
!--------------------------------------------------------------------------
!! gnuplot file(s) - optional

  call clock_start(33,'gnuplot time')

  select case(param%control)
  case default
     if(plot%gnu) then
        call gfile_init(trim(file%gnu),'file header',nprint)
        call bigobj_writeg(bigobj,'selector',nprint)
        call gfile_close
     end if
  end select

  call clock_stop(33)
!--------------------------------------------------------------------------
!! output file

  call clock_start(34,'outfile_init time')
  call bigobj_initwrite(fileroot)
  call bigobj_write(bigobj)
  call bigobj_closewrite()
  call clock_stop(34)
!--------------------------------------------------------------------------
!! cleanup and closedown

  call bigobj_delete(bigobj)

  call clock_stop(1)
  call clock_summary

  call log_close
  call clock_delete
!--------------------------------------------------------------------------

end program p_qprog
\end{verbatim}
\normalsize


\clearpage


%\chapter{doxygen Documentation for Sample Program}\label{sec:appb}
%
%This is now a separate document, ref~\cite{QPROGegdoc}, available on request, together
%with the source code.
% To generate files for including here, start with file.f90 and
% (1) doxygen file.f90
% (2) view file.html using firefox
% (3) print to ps file file.ddd.ps
% (4) use items (2)-(4) as described at start of previous subsection to
% get eps or pdf files as appropriate
   


%\section{template\_m}
%\label{sec:templatef90}
%\enlargethispage*{30pt} 
%\fbox{\includegraphics[width=0.99\textwidth]{template_m_docp1}~~}
%\clearpage
%
% \fbox{\includegraphics[width=0.99\textwidth]{template_m_docp2}~~}
%\clearpage
%
% \fbox{\includegraphics[width=0.99\textwidth]{template_m_docp3}~~}
%
%\section{oo\_m}
%\label{sec:oof90}
% 
%\fbox{\includegraphics[width=0.99\textwidth]{oo_m_docp1}~~}
%
%\clearpage
%
%
%\fbox{\includegraphics[width=0.99\textwidth]{oo_m_docp2}~~}

\chapter*{Acknowledgement}\label{sec:ackn}
%The efforts of Drs Morgan and Eastwood of Culham Electromagnetics Ltd in
%producing an early version of this document are gratefully acknowledged.
This work has been part funded by the RCUK Energy Programme under grant number
EP/I501045.
To obtain further information on the data and models underlying this
paper please contact PublicationsManager@ccfe.ac.uk.
The source code for the \coden\ program and its doxygen documentation
are also available on request.

%
\normalsize
%\chapter*{References}
%\bibliographystyle{unsrt}
%\bibliography{new,waynes,misc,warv,neuts,dg1srt}
%\input{annexa}
\end{document}
